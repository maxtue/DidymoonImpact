- A very good explanation of the theory behind the Miluphcode code can be found in \cite{Schaefer_2016} and \cite{Schaefer_2020}. The explanations in these two papers are concise, well written and difficult to improve upon. The aim of this theory section is therefore only to present the subset of these equations that were used for  the particular solid model used in this study. With the given equations, all parameters in Table \ref{tab:material_parameters} of Section \ref{sect:material_parameters} should be explained.

\subsection{Material derivative}
- In SPH we want to know the local velocity change at each particle location
- Show that:
$ \frac{dv_i}{dt} = \frac{\partial v}{\partial t} + v \cdot \nabla v $

- $v \cdot \nabla v $ holds for any scalar field property y that a particle traverses with velocity u $u \cdot \nabla y $ since it is the projection $u \cdot \nabla y = |u|\cdot|\nabla y| \cdot cos (\Theta)$ where $\Theta$ is the angle between $u$ and $\nabla y$.
- Thus, the local change in the field property $\frac{dy_i}{dt}$ at the point i in the Lagrangian view corresponds to the sum of the change of the field property at a point i over time $\frac{\partial y_i}{\partial t}$ and the change $u_i \cdot \nabla y_i $ of the field property due to the movement of the particle along the field with velocity $u$ in the Eulerian view:
$ \frac{dy_i}{dt} = \frac{\partial y_i}{\partial t} + u_i \cdot \nabla y_i $

This expression is called the Material Derivative $\frac{Dy}{dt}$:
$ \frac{Dy}{dt} \equiv \frac{\partial y}{\partial t} + (u \cdot \nabla y $

\begin{equation}
    {\frac {\mathrm {D} \mathbf {F} }{\mathrm {D} t}}={\frac {\partial \mathbf {F} }{\partial t}}+\left(\mathbf {u} \cdot \nabla \right)\mathbf {F}
\end{equation}

It gives us the change of the field property in the reference frame of the particle (denoted by the subscript i):

- In next sections $\frac{d}{dt}$ denotes material derivative and greek letters are used to denote spacial coordinates which go from 1 to 3. Einstein summation convention is used
\subsection{Conservation equations}

\subsubsection{Conservation of mass}
- continuity equation
\begin{equation}
    \frac{\mathrm {d}\rho }{\mathrm {d}t} + \rho \frac{\partial v^{\alpha} }{\partial x^\alpha} = 0
\end{equation}
- $\rho$ velocity, $v^{\alpha}$ material velocity

\subsubsection{Conservation of momentum}

\begin{equation} \label{eq:momentum}
    \frac{\mathrm {d} v^{\alpha} }{\mathrm {d}t} - \frac{1}{\rho} \frac{\partial \sigma^{\alpha \beta} }{\partial x^{\beta}} = 0
\end{equation}
- $\sigma$ stress tensor
\begin{equation}
    \sigma^{\alpha \beta} = - p\delta^{\alpha \beta} + S^{\alpha \beta}
\end{equation}
- p Druck, $\delta^{\alpha \beta}$ Kronecker Delta, $S^{\alpha \beta}$ deviatoric stress tensor

\subsubsection{Conservation of energy}
\begin{equation}
    \frac{\mathrm {d} u }{\mathrm {d}t} + \frac{p}{\rho} \frac{\partial v^{\alpha} }{\partial x^\alpha} - \frac{1}{\rho} S^{\alpha \beta} \dot{\epsilon}^{\alpha \beta} = 0
\end{equation}
- u internal energy, $\dot{\epsilon}$ strain rate tensor
\begin{equation}
    \dot{\epsilon}^{\alpha \beta} = \frac{1}{2} \left(\frac{\partial v^{\alpha} }{\partial x^\beta} + \frac{\partial v^{\beta} }{\partial x^\alpha}\right)
\end{equation}

- Set of Odes can be solved with integration schemes, only remaining unknown quantities at each step are pressure p and deviatoric stress tensor $S^{\alpha \beta}$
- deviatoric stress tensor $S^{\alpha \beta}$: Strength model
- pressure: Equation of State + porosity model

\subsection{Plasticity model}

\subsubsection{Constitutive equations}
- the deviatoric stress tensor is the main aspect of modeling a solid, for liquids it is 0.
- Constitutive equations describe time evolution of S
- elastic and plastic regime:
- this form of const equation takes care of elastic regime based on Hookes Law
- plastic regime modeled by damage and strength model
\begin{equation}
    \frac{\mathrm {d}S^{\alpha \beta} }{\mathrm {d}t}  = 2\mu \left(\dot{\epsilon}^{\alpha \beta} - \frac{1}{3}\delta^{\alpha \beta}\dot{\epsilon}^{\gamma \gamma}\right) + S^{\alpha \gamma}R^{\gamma \beta} - R^{\alpha \gamma}S^{\gamma \beta}
\end{equation}
- $\mu$ shear modulus as in Table \ref{tab:material_parameters}, rotation rate sensor $R^{\alpha \beta}$
\begin{equation}
    R^{\alpha \beta} = \frac{1}{2} \left(\frac{\partial v^{\alpha} }{\partial x^\beta} - \frac{\partial v^{\beta} }{\partial x^\alpha}\right)
\end{equation}

\subsubsection{Fracture model} \label{sect:fracture_model}
If enough force is applied, brittle materials can break. Cracks develop and get larger when tensile loading happens. SPH particles in this model carry both a quantity d for damage and a varying number of crack thresholds. The damage $d \in \left[0, 1\right]$ leads to a reduction of the influence of the deviatoric stress tensor $S^{\alpha \beta}$ on the stress tensor $\sigma^{\alpha \beta}$ and thereby on the momentum transfer in Equation \ref{eq:momentum}.

\begin{equation}
    \sigma^{\alpha \beta}_d = -\hat{p}\delta^{\alpha \beta} + (1 - d)S^{\alpha \beta}
\end{equation}

Also, damaged material will not be able to experience negative pressure:
\begin{equation}
    \hat{p} =
    \begin{cases}
        p,        & \text{for } p \geq 0 \\
        (1 - d)p, & \text{for } p < 0
    \end{cases}
\end{equation}
Negative pressure in this context describes tension which arises when solids are pulled. A damaged material will no longer exert any tension against pulling forces. \\

The starting point for accumulating damage are cracks in the material. These cracks are modeled as strain thresholds that are randomly distributed through the material. The threshold values for all strains $j \in 1...n$ in a given Volume $V$ are determined by the Weibull distribution:

\begin{equation}
    \epsilon = \left(\frac{j}{kV}\right)^\frac{1}{m}
\end{equation}

The material parameters $k$ and $m$ for the weibull distribution have been experimentally determined for basalt by \cite{Nakamura_2007}. They can be found in Table \ref{tab:material_parameters} in Section \ref{sect:material_parameters}.

Whenever the strain on a Volume exceeds its lowest crack threshold, damage will grow around this crack as

\begin{equation}
    \frac{\mathrm{d}}{\mathrm{d}t} d^{\frac{1}{3}} = \frac{c_g}{R}
\end{equation}

where $R$ is the distance to the crack and $c_g$ the growth velocity which follows from the density $\rho$, bulk modulus $K_0$, current damage $d$ and shear modulus $\mu$:

\begin{equation}
    c_g = 0.4\frac{1}{\rho}\sqrt{K_0 + \frac{4}{3}(1-d)\mu}
\end{equation}

\subsubsection{Pressure dependent yield strength}
- ideas in \cite{Collins_2004}
- implementation in \cite{Jutzi_2015}
- Transition from linearity in elastic case to the plastic case
- Von Mises model limits the deviatoric stress at a constant threshold.

- Main invariants of a tensor are invariant under base changes
- They can therefore be used for sph particles regardless of their motion and relative orientation
- $J_2$ is the second main invariant of $S^{\alpha \beta}$ calculated as:
\begin{equation}
    J_2 = \frac{1}{2}S^{\alpha \beta}S_{\alpha \beta}
\end{equation}

It is used to compute the limiting factor $f_Y$ for each particle:

\begin{equation}
    f_Y = min \left[\frac{Y}{\sqrt{J}}, 1 \right]
\end{equation}

\begin{equation}
    S^{\alpha \beta} \rightarrow f_Y S^{\alpha \beta}
\end{equation}

Y is the yield strength and can be determined in different ways. In the simplest case of a von Mises strength model it is constant. In a more sophisticated model developed by \cite{Collins_2004} and later improved by \cite{Jutzi_2015} Y depends upon the pressure and the damage from the fracture model introduced in Section \ref{sect:fracture_model}. In this approach, the yield strength is a linear combination of the yield strength for intact material $Y_I$ and for completely damaged material $Y_D$.

\begin{equation}
    Y = \left(1 - D \right)Y_I + DY_D
\end{equation}

\begin{equation}
    Y_I = Y_0 + \frac{\mu_I p}{1 + \frac{\mu_I p}{Y_M - Y_0}}
\end{equation}
\begin{equation}
    Y_D = \mu_D p
\end{equation}

The coefficients of internal friction $\mu = tan(\alpha)$ depend upon the angles $\alpha_{intact}$ and $\alpha_{damaged}$. $Y_0$ is the cohesion or shear strength at $p = 0$ and $Y_M$ the shear strength at $p = \infty$.


\subsection{Equation of State}
- calculating pressure p for high velocity impacts
- \cite{Tillotson_1962}

1) $u \leq E_{iv}$
\begin{equation}
    p = \left(a_T + \frac{b_T}{1 + \frac{u}{E_0\eta^2}}\right)\rho u + A_T \chi + B_T \chi^2
\end{equation}


2) $u \geq E_{cv}$
\begin{equation}
    p = a_T \rho u + \left( \frac{b_T \rho u}{1 + \frac{u}{E_0 \eta^2}} + A_T \chi \cdot exp\left[-\beta_T\left(\frac{\rho_0}{\rho} - 1\right)\right] \right) \cdot exp \left(-\alpha_T \left[\frac{\rho_0}{\rho} - 1 \right]^2 \right)
\end{equation}


with $eta = \frac{\rho}{\rho_0} $and$\chi = eta - 1$. $rho_0, A_T, B_T, E_0, E_{iv}, E_{cv}, a_T, b_T, \alpha_T and \beta_T$ are material dependent parameters (see Table \ref{tab:material_parameters} in Section \ref{sect:material_parameters}).

3) $u$ between $E_{iv}$ and $E_{cv}$: linear interpolation


\subsection{Porosity}
There are different ways in which porosity can be modeled depending on the pore size. Macro porosity with pore sizes above the resolution of the simulation can be accounted for in the initial conditions. This however becomes impossible for granular material with sub-resolution sized grains and pores.

Microporosity models porosity as an additional material property and can be applied independently of the resolution.

The main goal of the porosity model is to separate the compression of the pores in a porous material from the elastic or plastic deformation that is modeled with the strength model.

In these simulations, a microporosity p-$\alpha$ model as outlined in \cite{Jutzi_2008} is used. The distention $\alpha \in \left[1,\infty\right)$ relates the current density $\rho$ to the solid density $\rho_s$ which is reached if the material is fully compressed. For a non-porous material $\alpha$ equals one.

\begin{equation}
    \alpha \equiv \frac{\rho_s}{\rho} = \frac{1}{1 - \phi}
\end{equation}

with the porosity $\phi \in \left[0, 1\right)$.

The p-$\alpha$ model connects the distention with the pressure. Empirically it was found \cite{Carroll_1972} that this relation can be modeled as

\begin{equation}
    p = \frac{1}{\alpha}\,p_s(\alpha\rho, u)
\end{equation}

where $p_s$ is the pressure of the material in the solid, non-porous case. At the same time the distention depends on the pressure. There are different ways to model this relation. In this study, a crush-curve with quadratic dependence on the pressure has been used

\begin{equation}
    \alpha = 1 + \left(\alpha_e - 1 \right) \frac{\left(p_s - p\right)^2}{\left(p_s - p_e\right)^2}
\end{equation}

where $\alpha_e$ = $\alpha\left(p_e\right)$ and $p_e$ is the pressure at which the material enters the plastic from the elastic regime. In the elastic regime below $p_e$ as well as above the second threshold $p_s$ the distention stays constant. At $p_s$ the material is completely solid with regard to porosity and the pores can not be compacted anymore. When calculating the change of distention in the elastic regime, the dependence of the distention on the pressure has to be accounted for. Using the chain rule

\begin{equation}
    \dot{\alpha} = \frac{\mathrm{d}\alpha}{\mathrm{d}p}\frac{\mathrm{d}p}{\mathrm{d}t}
\end{equation}

one arrives at the following expression for the time evolution of the distention:

\begin{equation}
    \left(\frac{\mathrm{d}\alpha}{\mathrm{d}p}\right)_{elastic} = \frac{\alpha^2}{K_0}\left(1 - \left[ \frac{1}{h(\alpha)^2}[\right]\right)
\end{equation}

\begin{equation}
    h(\alpha) = 1 + (\alpha - 1)\frac{c_e - c_0}{c_0(\alpha_e - 1)}
\end{equation}

where $c_0$ and $c_e$ are the sound speeds for material that is fully expanded and at the begin of the elastic regime respectively.
