\subsection{Conservation laws}
Momentum:
\frac{\mathrm d \vec{v}_{a}}{\mathrm dt} = - \frac{1}{\rho_{a}} \sum \limits_{b} m_{b} \frac{P_{b}}{\rho_{b}} \nabla_{a} W_{ab}

\subsection{Artificial viscosity}

{\displaystyle {\frac {\mathrm {d} {\vec {v}}_{a}}{\mathrm {d} t}}=-\sum \limits _{b}m_{b}\left({\frac {P_{b}}{\rho _{b}^{2}}}+{\frac {P_{a}}{\rho _{a}^{2}}}+\Pi _{ab}\right)\nabla _{a}W_{ab}}


Results
- include cratering for 45 degree
- explain calculation of beta factor
- along impact direction also for 45 degree
- particles near surface not important because of lorenz curve, every particle has constant mass(?)
-

Discussion:
- why does beta rise again for Y=1MPa?
- no rise for 45 degree impact
- overall a lot less difference between strengths and porosities than other research groups
